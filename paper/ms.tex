\documentclass{aastex62}

\usepackage{graphicx}
\usepackage{gensymb}
\usepackage{bm}
\usepackage{amsmath}

\newcommand{\Gus}[1]{\textcolor{red}{#1}}

\newcommand{\Msun}{\ensuremath{\text{M}_\odot}}
\newcommand{\pc}{\ensuremath{\text{pc}}}
\newcommand{\kpc}{\ensuremath{\text{kpc}}}
\newcommand{\Mpc}{\ensuremath{\text{Mpc}}}
\newcommand{\Myr}{\ensuremath{\text{Myr}}}
\newcommand{\Gyr}{\ensuremath{\text{Gyr}}}
\newcommand{\SFR}{\ensuremath{\text{SFR}}}

\newcommand{\GHz}{\ensuremath{\text{GHz}}}

\newcommand{\CI}{\ensuremath{\text{C~\textsc{i}}}}
\newcommand{\CII}{\ensuremath{\text{C~\textsc{ii}}}}
\newcommand{\OIII}{\ensuremath{\text{O~\textsc{iii}}}}
\newcommand{\CO}{\ensuremath{\text{CO}}}

\newcommand{\abs}[1]{\left| #1 \right|}
\newcommand{\uth}{\textsuperscript{th}}
\newcommand{\n}{\text{n}}

\newcommand{\beq}{\begin{equation}}
\newcommand{\eeq}{\end{equation}}

\newcommand{\ps}[1]{\ensuremath{P_{#1,#1}}}
\newcommand{\xps}[2]{\ensuremath{P_{#1,#2}}}

\newcommand{\kmax}{\ensuremath{k_{\text{max}}}}

\newcommand{\mynu}[2]{\ensuremath{\nu_{#1,\text{#2}}}}
\newcommand{\denps}{\ensuremath{P_{\delta,\delta}}}
\newcommand{\pstot}[1]{\ensuremath{P_{#1,\text{tot}}}}
\newcommand{\Bsum}{\ensuremath{\tilde{B}}}

\newcommand{\E}[1]{\mathrm{E}[#1]}
\newcommand{\Var}[1]{\mathrm{Var}[#1]}
\newcommand{\Cov}[2]{\mathrm{Cov}[#1,#2]}
\newcommand{\avg}[1]{\ensuremath{\langle #1 \rangle}}
\newcommand{\SN}{\ensuremath{\text{S}/\text{N}}}

\newcommand{\penn}{Department of Physics \& Astronomy, University of
Pennsylvania, 209 South 33rd St., Philadelphia, PA 19104, USA}

\begin{document}

\title{Multi-line Intensity Mapping}

\email{abeane@sas.upenn.edu}

\author{Angus Beane}
\affil{\penn}

\author{Adam Lidz}
\affil{\penn}

\author{others}

\shorttitle{Multi-line Intensity Mapping}
\shortauthors{Beane et al.}

\begin{abstract}
Abstract.
\end{abstract}

\section{Introduction}\label{sec:intro}

\section{Three Field Approach}\label{sec:tf}
We begin by considering the three field approach. We first start with this
relatively simple case to gain some intuition for the \SN{} behavior when the
cross-spectra of multiple lines are combined.

We are concerned with the problem of estimating the power spectrum of each
line only from the three cross-spectra. To be more concrete, we model each
auto-spectrum and cross-spectrum as,
\beq\label{eq:auto_xps}
\begin{split}
\ps{i}(k,z) &= B_i(z)^2 \denps(k,z) \\
\xps{i}{j}(k,z) &= B_i(z) B_j(z) \denps(k,z)\text{,}
\end{split}
\eeq
where $B_i(z) \equiv b_i(z) \avg{I_i}(z)$. We will generally suppress the
redshift and $k$ arguments for brevity. We can model the auto-spectra and
cross-spectra in this way only on sufficiently large scales, i.e. only when
$k<\kmax$ for some value of $k_{\text{max}}$. The value of \kmax{} is set by
two conditions. First, we must have that linear biasing is a good description
for each field $i$. This way $B_i$ for each line will be independent of $k$.
Second, for each pair of lines $(i,j)$ we must have that $\abs{r_{i,j}(k)} =
1$ when $k<\kmax$.

Eventually it will be desirable to place constraints on $\denps(k)$. For now,
however, we simply wish to understand how well future experiments will be able
to determine each of $B_i$ as a function of $z$. We can proceed simply by
computing the relevant Fisher matrix:
\beq\label{eq:fisher_tf}
F_{i,j} = 
\int \frac{\text{d}k\,k^2}{2\pi^2} V_{\text{surv}} 
\sum_{\substack{l,l'\\l > l'}} \sum_{\substack{m,m'\\m > m'}}
\frac{\partial \hat{P}_{l,l'}(k)}{\partial B_i}
\frac{1}{\Cov{\hat{P}_{l,l'}(k)}{\hat{P}_{m,m'}(k)}} 
\frac{\partial \hat{P}_{m,m'}(k)}{\partial B_j}\text{,}
\eeq
where the sum of $l$, $l'$, $m$, and $m'$ is over each of the three lines.
Equation~\ref{eq:fisher_tf} can be readily modified to accomodate the case of
more than three lines (by extending the summations to include these lines) and
to accomodate the auto-spectra (by allowing $l=l'$ and $m=m'$ in the relevant
summations). We will consider these cases in Sections~\Gus{x}~and~\Gus{y},
respectively.

For the variance and covariance of the power spectra, it can be shown that,
\beq\label{eq:var_cov}
\begin{split}
\Var{\xps{i}{j}} &= \xps{i}{j}^2 + \pstot{i}\pstot{j} \\
\Cov{\xps{i}{j}}{\xps{i}{k}} &= \pstot{i}\xps{j}{k} +
\xps{i}{j}\xps{i}{k}\text{,}
\end{split}
\eeq
where $\pstot{i} \equiv \ps{i} + N_i$. Note that $N_i$ contains contributions
from instrument noise, residual foreground contamination, and
interloping/extraloping lines. While computing the Fisher matrix numerically
is trivial, we will first work out the noise-dominated case which has a
somewhat tractable analytic solution.


\bibliography{references}

\end{document}
