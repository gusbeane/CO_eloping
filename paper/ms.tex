\documentclass{aastex62}

\usepackage{graphicx}
\usepackage{gensymb}
\usepackage{bm}
\usepackage{amsmath}

\newcommand{\Gus}[1]{\textcolor{red}{#1}}

\newcommand{\Msun}{\ensuremath{\text{M}_\odot}}
\newcommand{\pc}{\ensuremath{\text{pc}}}
\newcommand{\kpc}{\ensuremath{\text{kpc}}}
\newcommand{\Mpc}{\ensuremath{\text{Mpc}}}
\newcommand{\Myr}{\ensuremath{\text{Myr}}}
\newcommand{\Gyr}{\ensuremath{\text{Gyr}}}
\newcommand{\SFR}{\ensuremath{\text{SFR}}}

\newcommand{\GHz}{\ensuremath{\text{GHz}}}

\newcommand{\CI}{\ensuremath{\text{C~\textsc{i}}}}
\newcommand{\CII}{\ensuremath{\text{C~\textsc{ii}}}}
\newcommand{\OIII}{\ensuremath{\text{O~\textsc{iii}}}}
\newcommand{\CO}{\ensuremath{\text{CO}}}

\newcommand{\abs}[1]{\left| #1 \right|}
\newcommand{\uth}{\textsuperscript{th}}
\newcommand{\n}{\text{n}}

\newcommand{\beq}{\begin{equation}}
\newcommand{\eeq}{\end{equation}}

\newcommand{\mynu}[2]{\ensuremath{\nu_{#1,\text{#2}}}}
\newcommand{\denps}{\ensuremath{P_{\delta,\delta}}}
\newcommand{\xps}[2]{\ensuremath{P_{#1,#2}}}
\newcommand{\pstot}[1]{\ensuremath{P_{#1,\text{tot}}}}
\newcommand{\Bsum}{\ensuremath{\tilde{B}}}

\newcommand{\E}[1]{\mathrm{E}[#1]}
\newcommand{\Var}[1]{\mathrm{Var}[#1]}
\newcommand{\Cov}[2]{\mathrm{Cov}[#1,#2]}
\newcommand{\avg}[1]{\ensuremath{\langle #1 \rangle}}

\newcommand{\penn}{Department of Physics \& Astronomy, University of
Pennsylvania, 209 South 33rd St., Philadelphia, PA 19104, USA}

\begin{document}

\title{CO Interlopers Eloping from [C~\textsc{ii}] Surveys}

\email{abeane@sas.upenn.edu}

\author{Angus Beane}
\affil{\penn}

\author{Adam Lidz}
\affil{\penn}

\author{others}

\begin{abstract}
Intensity mapping stands to be a promising probe of the large-scale structure
of the universe at high redshift ($z>3$). The experiment TIME, currently
undergoing a preliminary engineering run, will measure [\CII] emission from
the redshift ranges $5<z<9$ {\bf check}. An important problem to overcome will
be to remove interloping CO lines. These are lines that are emitted at a
different frequency than [\CII] but are observed at the same frequency from
gas residing at a different redshift. The measurement of [\CII] will be
impacted by emission from three CO transitions
\end{abstract}

\section{Introduction}\label{sec:intro}
At high redshift ($z>3$) the most promising probe of the large-scale structure
(LSS) of the universe is line intensity mapping. Rather than attempting to
resolve individual galaxies, intensity mapping aims to measure the total
emission from an atomic or molecular line of all galaxies within a rather
coarse pixel. Commonly proposed lines include H$\alpha$, H$\beta$, Ly$\alpha$,
[\CII], [\OIII], and 21~cm. During the Epoch of Reionization (EoR;
$z\gtrsim6$), the hydrogen lines also measure the bulk emission from neutral
hydrogen residing in the diffuse intergalactic medium (IGM).

\section{Motivation}\label{sec:theory}
\subsection{Interloper Problem}\label{ssec:interloper}
The primary data is the observed intensity in a collection of voxels. In
practice, one will make measurements in sky coordinates and frequency space,
but to simplify the discussion we consider the observation of a finite
comoving box with volume $V$. The box is small enough that if one length is
assumed to lie along frequency space that the corresponding redshift extent of
that axis is small enough that the observed voxels can be considered
co-evolving. We are then interested in measuring $I_{l}(\bm{x})$, or the
collective emission from line $l$ from each voxel. In order to understand the
statistical properties of $I(\bm{x})$, we can compute the power spectrum which
is related to the Fourier transform of $I$,
\beq
P_{l,l}(k) = \avg{I_l(\bm{k}) I_l^{*}(\bm{k})}\text{.}
\eeq
Since the line intensity is a biased tracer of the underlying density field,
it is useful to parametrize the power spectrum as,
\beq
P_{l,l}(k) = b_l^2 \avg{I_l}^2 \denps(k)\text{,}
\eeq
where $\denps(k)$ is the matter density power spectrum. Measuring
$b_l$ and $\avg{I_l}$ independently is a useful endeavour, but not the focus
of our study. Therefore, we will simplify and define $B_l \equiv b_l
\avg{I_l}$, such that,
\beq
P_{l,l}(k) = B_l^2 \denps(k)
\eeq

Suppose we have a line $l$ which we are attempting to measure emission from at
some target redshift $z_l$ and some observed frequency $\mynu{l}{o} =
\mynu{l}{e}/(1+z_l)$. Then, the interloper problem is that at various other
redshifts $z_i$ for $i=1,2,3,...$ there will be emission from lines with
emitted frequencies at redshift $z_i$ that satisfy the constraint $\mynu{l}{o}
= \mynu{i}{o}$, or equivalently,
\beq
\mynu{i}{e} = \mynu{l}{e}\frac{1+z_i}{1+z_l}\text{.}
\eeq

Assuming perfect foreground removal, if one makes a naive power spectrum
estimate of the observed intensity at a frequency $\mynu{l}{o}$, then the
observed power spectrum will be,
\beq\label{eq:observed_with_int}
P_{\mynu{l}{o}} = P_{l} + \sum_{i} P_{i}\text{.}
\eeq
Since $P_l$ is the object of interest, we must robustly measure and subtract
each of $P_i$.

\subsection{Three-Field Approach}\label{ssec:three_field}

Consider one of the interloping lines, and assume that there exists two other
lines emitted at the same redshift but by virtue of having a different
emission frequency have a different observed frequency. For example, CO(6-5)
emitted at $z=1$ will be observed at $345\,\GHz$, while CO(5-4) would be
observed at $288\,\GHz$ and CO(4-3) at $230\,\GHz$. As long as these
frequencies are in the observing band of some instrument, then we have enough
freedom to directly constrain the auto-spectrum of each line. We could proceed
as we did previously, forming suitable products and ratios to measure the
relevant cross-spectra \citep{2018arXiv181110609B},
\beq\label{eq:old_way}
P_{\text{CO(6-5),CO(6-5)}} = 
\frac{ \xps{\text{CO(6-5)}}{\text{CO(5-4)}} 
       \xps{\text{CO(4-3)}}{\text{CO(6-5)}} }
     { \xps{\text{CO(5-4)}}{\text{CO(4-3)}}}\text{.}
\eeq
However, this is a bit unconventional and extending to more than three lines
is cumbersome.

Now consider an arbitrary triple of the interloping lines $(i, j, k)$ where
none of $i$, $j$, or $k$ are equal (and in particular none of $z_i$, $z_j$, or
$z_k$ are equal). Now consider the case of removing emission from line $i$,
which is interloping at redshift $z_i$. One can cross-correlate line $i$ with
line $j$ if the frequency at which line $j$ is emitted at redshift $z_i$ is
within the observed frequency band. That is, we'd like to observe line $j$ at
a different frequency $\mynu{j}{o}'$ such that the emitted redshift $z_j'$ is
equal to $z_i$, so that we can cross-correlate the two. In other words, we
need $\mynu{j}{o}'$ to satisfy,
\beq\label{eq:nu_obs}
\begin{split}
\mynu{j}{o}' &= \mynu{i}{o} \frac{\mynu{j}{e}}{\mynu{i}{e}} \\
&= \mynu{i}{o} \frac{1+z_j}{1+z_i}\text{.}
\end{split}
\eeq
If the frequency $\mynu{j}{o}'$ is in the observing band of an instrument
(though in a different bin than $\mynu{i}{o}$), then a cross-corelation
between line $i$ and line $j$ can be made. This can be used to measure the
product $B_iB_j$. However, this does not allow us to remove either line $i$ or
line $j$ reliably, since the relevant quantity to subtract from
Equation~\ref{eq:observed_with_int} is $B_i^2 + B_j^2$, for which a direct
constraint cannot be made. Perhaps measurements of the cross-bispectrum may be
possible and allow us to measure $B_i$ and $B_j$ independently
\citep{2018ApJ...867...26B}, but measuring cross-bispectra is difficult in
practice and a reliable cross-spectrum measurement will probably be possible
with less survey area.

However, in the case that $\mynu{k}{o}'$ (defined similarly to
Equation~\ref{eq:nu_obs}) also lies in the observing band, then a three-field
approach can be employed to directly measure each of $B_i^2$, $B_j^2$, and
$B_k^2$.

One could proceed as we did previously \citep{2018arXiv181110609B}, estimating
the auto-spectra in each line by forming suitable products and ratios of the
various cross-spectra. However, we instead take a slightly different approach
here by performing a joint fit for all of the various auto-spectra. First, we
assume a measurement of each of $\xps{i}{j}$. Then the relevant variance and
covariances are \citep[e.g.][]{2015JCAP...03..034V},
\beq
\begin{split}
\Var{\xps{i}{j}} &= \xps{i}{j}^2 + \pstot{i}\pstot{j} \\
\Cov{\xps{i}{j}}{\xps{i}{k}} &= \pstot{i}\xps{j}{k} +
\xps{i}{j}\xps{i}{k}\text{.}
\end{split}
\eeq
We are then making a measurement of each of $\xps{i}{j}$, $\xps{j}{k}$,
and $\xps{k}{i}$. We further model each of these power spectra as,
\beq
\xps{i}{j}(k) = B_i(k) B_j(k) \denps(k)\text{.}
\eeq
That is, we are only working in the regime where $r_{i,j}=1$. What this
translates to is that we may only perform this interloper removal on
sufficiently large scales, i.e. where $k<k_{\text{max}}$, where
$k_{\text{max}}$ is set by wherever the lines $i$ and $j$ start to
decorrelate. The value of $k_{\text{max}}$ is the only aspect of our analysis
that is model dependent.

\section{CO Measurements} \label{sec:co_measurements}
\subsection{CO Modelling} \label{ssec:co_modelling}
To model the statistical fluctuation and amplitude of the \CO{} intensity maps
we follow \citet{2016ApJ...825..143L}, and briefly summarize the procedure
here. We begin by assuming a Schechter form for the star formation rate
function,
\beq\label{eq:schechter}
\phi(\SFR)\,\text{d}\SFR = \phi_*
\left(\frac{\SFR}{\SFR_*}\right)^{\alpha}\exp{\left[-\frac{\SFR}{\SFR_*}\right]}
\frac{\text{d}\SFR}{\SFR_*}
\eeq
where $\alpha$ is the faint-end slope, $\SFR_*$ is the characteristic
star-formation rate, and $\phi_*$ is the characteristic number density. The
average specific intensity in a line $l$ can be related to the comoving
emissivity by \citep{2011ApJ...741...70L, 2013ApJ...768...15P},
\beq\label{eq:emiss_to_int}
\avg{I_l} = \frac{\epsilon_l}{4\pi \nu_{\text{rest},l}}\frac{c}{H(z)}\text{,}
\eeq
where $\nu_{\text{rest},l}$ is the rest-frame emission frequency and
$\epsilon_l$ is the comoving emissivity, each of line $l$. Assuming a linear
luminosity-SFR relation with a constant of proportionality of $L_0^l$, along
with the Schechter form for the SFR function (Equation~\ref{eq:schechter}), it
follows that \citep{2013ApJ...768...15P},
\beq\label{eq:com_emiss}
\epsilon_l = \phi_* L_0^{l}
\frac{\SFR_*}{1\,\Msun\text{yr}^{-1}}\Gamma{(2+\alpha)}\text{.}
\eeq
We adopt the value of $L_0^{\CO}$ for each line transition from
\citet{2010JCAP...11..016V}.

\subsection{Something} \label{ssec:co_}
We now consider the prospects of measuring the \CO{} transition using a
multi-field approach. First, we choose to label each line with the upper
transitional level, e.g. CO(3-2) is referred to as ``3.'' The relevant Fisher
matrix is then,
\beq\label{eq:co_fisher}
F_{i,j} = 
\int \frac{\text{d}k\,k^2}{2\pi^2} V_{\text{surv}} 
\sum_{\substack{l,l'\\l > l'}} \sum_{\substack{m,m'\\m > m'}}
\frac{\partial \hat{P}_{l,l'}(k)}{\partial B_i}
\frac{1}{\Cov{\hat{P}_{l,l'}(k)}{\hat{P}_{m,m'}(k)}} 
\frac{\partial \hat{P}_{m,m'}(k)}{\partial B_j}\text{,}
\eeq
where $i$ and $j$ each refer to a line by its upper transitional level and the
$l$, $l'$, $m$ and $m'$ sum over all lines considered.

Ignoring interlopers and foreground contamination for a moment, we can
simplify this expression as (see Appendix~\ref{app:fisher_deriv} for a
derivation):
\beq\label{eq:fisher_nocontam}
\begin{split}
F_{i,j} =&
\int \frac{\text{d}k\,k^2}{2\pi^2} V_{\text{surv}} \denps^2(k) \\
\times & \left[ 
\delta_{ij} \left( 
\sum_{l\neq i} \frac{B_l^2}{\xps{i}{l}^2 + \pstot{i}\pstot{l}}
+ \frac{1}{\pstot{i}\xps{\delta}{\delta} + \xps{i}{\delta}^2}
\right) \right. \\
&\left. + (1-\delta_{ij}) \left( 
\sum_{l\neq i,j} \frac{B_l^2}{\pstot{l}\xps{i}{j} + \xps{i}{l}\xps{l}{j}}
+\frac{B_i}{\pstot{i}\xps{j}{\delta} + \xps{i}{\delta}\xps{i}{j}}
+ \frac{B_j}{\pstot{j}\xps{i}{\delta} + \xps{j}{\delta}\xps{i}{j}}
+ \frac{B_iB_j}{\xps{i}{j}^2 + \pstot{i}\pstot{j}}
\right)
\right]\text{.}
\end{split}
\eeq
Including the auto-spectra in this analysis simply requires summing over all
$l$ values, replacing $\xps{i}{i}$ with $\pstot{i}$ \Gus{I think this is
correct?}.

\appendix

\section{Fisher Matrix Derivation}\label{app:fisher_deriv}
We show a derivation of Equation~\ref{eq:fisher_nocontam}. First, consider the
Fisher matrix from Equation~\ref{eq:co_fisher}.
\beq\label{eq:co_fisher_2}
F_{i,j} = 
\int \frac{\text{d}k\,k^2}{2\pi^2} V_{\text{surv}} 
\sum_{\substack{l,l'\\l > l'}} \sum_{\substack{m,m'\\m > m'}}
\frac{\partial \hat{P}_{l,l'}(k)}{\partial B_i}
\frac{1}{\Cov{\hat{P}_{l,l'}(k)}{\hat{P}_{m,m'}(k)}} 
\frac{\partial \hat{P}_{m,m'}(k)}{\partial B_j}\text{,}
\eeq
where $i$ and $j$ each refer to a line by its upper transitional level and the
$l$, $l'$, $m$ and $m'$ sum over all lines considered. We exploit the fact
that the Fisher matrix is unchanged under the substitution,
\begin{equation*}
\sum_{\substack{l,l'\\l > l'}} \rightarrow 
\frac{1}{2} \sum_{\substack{l,l'\\l \neq l'}}\text{.}
\end{equation*}
Next, we can write down the derivatives of the power spectra as,
\beq\label{eq:psderiv}
\frac{\partial \hat{P}_{l,l'}(k)}{\partial B_i}
= \left( B_{l'} \delta_{il} + B_l\delta_{il'} \right) \denps(k)\text{.}
\eeq
Now we can also write down the covariance formula as,
\beq\label{eq:cov}
\begin{aligned}
\left(\Cov{\hat{P}_{l,l'}}{\hat{P}_{m,m'}}\right)^{-1} = 
&(\pstot{l}\xps{l'}{m'} + \xps{l}{l'}\xps{l}{m'})^{-1} &&\times \delta_{lm} (1-\delta_{l'm'}) \\
&+ (\pstot{l}\xps{l'}{m} + \xps{l'}{l}\xps{l}{m})^{-1} &&\times \delta_{lm'} (1-\delta_{l'm}) \\
&+ (\pstot{l'}\xps{l}{m'} + \xps{l'}{l}\xps{l'}{m'})^{-1} &&\times \delta_{l'm} (1-\delta_{lm'}) \\
&+ (\pstot{l'}\xps{l}{m} + \xps{l'}{l}\xps{l'}{m})^{-1} &&\times \delta_{l'm'} (1-\delta_{lm}) \\
&+ (\xps{l}{l'}^2 + \pstot{l}\pstot{l'})^{-1} &&\times \delta_{lm} \delta_{l'm'} \\
&+ (\xps{l}{l'}^2 + \pstot{l}\pstot{l'})^{-1} &&\times \delta_{lm'} \delta_{l'm}\text{.}
\end{aligned}
\eeq
We are at liberty to distribute the inverse to all terms since upon inspection
of the Kronecker delta's one and only one term will contribute to the sum. In
the case that the covariance vanishes (e.g.
$\Cov{\hat{P}_{1,2}}{\hat{P}_{3,4}}=0$) we exclude this term from the sum in
Equation~\ref{eq:co_fisher_2}. Since the product,
\begin{equation*}
\frac{\partial \hat{P}_{l,l'}(k)}{\partial B_i}
% \frac{1}{\Cov{\hat{P}_{l,l'}(k)}{\hat{P}_{m,m'}(k)}} 
\frac{\partial \hat{P}_{m,m'}(k)}{\partial B_j}
\end{equation*}
is symmetric under $(l,l',m,m') \rightarrow (l',l,m',m)$\footnote{and the
summation is as well}, we can equivalently use the following formula for the
covariance:
\beq\label{eq:cov_simple}
\begin{aligned}
\left(\Cov{\hat{P}_{l,l'}}{\hat{P}_{m,m'}}\right)^{-1} = 
&4 (\pstot{l}\xps{l'}{m'} + \xps{l}{l'}\xps{l}{m'})^{-1} &&\times \delta_{lm} (1-\delta_{l'm'}) \\
&+ 2(\xps{l}{l'}^2 + \pstot{l}\pstot{l'})^{-1} &&\times \delta_{lm} \delta_{l'm'}\text{.}
\end{aligned}
\eeq
Combining these facts we can arrive at the following formula for the Fisher matrix:
\beq\label{eq:co_fisher_explicity}
\begin{split}
F_{i,j} = 
\frac{1}{4} \int \frac{\text{d}k\,k^2}{2\pi^2} V_{\text{surv}} 
\denps(k)^2
\sum_{\substack{l,l'\\l \neq l'}} \sum_{\substack{m,m'\\m \neq m'}}
&\left( B_{l'} \delta_{il} + B_l\delta_{il'} \right)
\left( B_{m'} \delta_{jm} + B_m\delta_{jm'} \right) \\
\times&\left( \frac{4}{\pstot{l}\xps{l'}{m'} + \xps{l}{l'}\xps{l}{m'}} \delta_{lm} (1-\delta_{l'm'})
+ \frac{2}{\xps{l}{l'}^2 + \pstot{l}\pstot{l'}} \delta_{lm} \delta_{l'm'}\right)
\text{.}
\end{split}
\eeq
This formula can be expanded and simplified to give Equation~\ref{eq:fisher_nocontam}.

\bibliography{references}

\end{document}
