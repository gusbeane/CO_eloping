\documentclass{aastex62}

\usepackage{graphicx}
\usepackage{gensymb}
\usepackage{bm}
\usepackage{amsmath}

\newcommand{\Gus}[1]{\textcolor{red}{#1}}

\newcommand{\Msun}{\ensuremath{\text{M}_\odot}}
\newcommand{\pc}{\ensuremath{\text{pc}}}
\newcommand{\kpc}{\ensuremath{\text{kpc}}}
\newcommand{\Mpc}{\ensuremath{\text{Mpc}}}
\newcommand{\Myr}{\ensuremath{\text{Myr}}}
\newcommand{\Gyr}{\ensuremath{\text{Gyr}}}

\newcommand{\CII}{C~\textsc{ii}}

\newcommand{\abs}[1]{\left| #1 \right|}
\newcommand{\uth}{\textsuperscript{th}}
\newcommand{\n}{\text{n}}

\newcommand{\beq}{\begin{equation}}
\newcommand{\eeq}{\end{equation}}

\newcommand{\mynu}[2]{\ensuremath{\nu_{#1,\text{#2}}}}
\newcommand{\denps}{\ensuremath{P_{\delta,\delta}}}
\newcommand{\xps}[2]{\ensuremath{P_{#1,#2}}}
\newcommand{\pstot}[1]{\ensuremath{P_{#1,\text{tot}}}}

\newcommand{\E}[1]{\mathrm{E}[#1]}
\newcommand{\Var}[1]{\mathrm{Var}[#1]}
\newcommand{\Cov}[2]{\mathrm{Cov}[#1,#2]}
\newcommand{\avg}[1]{\ensuremath{\langle #1 \rangle}}

\begin{document}

The primary data is the observed intensity in a collection of voxels. In
practice, one will make measurements in sky coordinates and frequency space,
but to simplify the discussion we consider the observation of a finite
comoving box with volume $V$. The box is small enough that if one length is
assumed to lie along frequency space that the corresponding redshift extent of
that axis is small enough that the observed voxels can be considered
co-evolving. We are then interested in measuring $I_{l}(\bm{x})$, or the
collective emission from line $l$ from each voxel. In order to understand the
statistical properties of $I(\bm{x})$, we can compute the power spectrum which
is related to the Fourier transform of $I$,
\beq
P_{l}(k) = \avg{I(\bm{k}) I^{*}(\bm{k})}\text{.}
\eeq
Since the line intensity is a biased tracer of the underlying density field,
it is useful to parametrize the power spectrum as,
\beq
P_l(k) = b_l^2 \avg{I_l}^2 \denps{P_{\delta}}(k)\text{,}
\eeq
where $\denps{P_{\delta}}$ is the matter density power spectrum. Measuring
$b_l$ and $\avg{I_l}$ independently is a useful endeavour, but not the focus
of our study. Therefore, we will simplify and define $B_l \equiv b_l
\avg{I_l}$, such that,
\beq
P_{l}(k) = B_l^2 \denps(k)
\eeq

Suppose we have a line $l$ which we are attempting to measure emission from at
some target redshift $z_l$ and some observed frequency $\mynu{l}{o} =
\mynu{l}{e}/(1+z_l)$. Then, the interloper problem is that at various other
redshifts $z_i$ for $i=1,2,3,...$ there will be emission from lines with
emitted frequencies at redshift $z_i$ that satisfy the constraint $\mynu{l}{o}
= \mynu{i}{o}$, or equivalently,
\beq
\mynu{i}{e} = \mynu{l}{e}\frac{1+z_i}{1+z_l}\text{.}
\eeq

Assuming perfect foreground removal, if one makes a naive power spectrum
estimate of the observed intensity at a frequency $\mynu{l}{o}$, then the
observed power spectrum will be,
\beq\label{eq:observed_with_int}
P_{\mynu{l}{o}} = P_{l} + \sum_{i} P_{i}\text{.}
\eeq
Since $P_l$ is the object of interest, we must robustly measure and subtract
each of $P_i$.

Now consider an arbitrary triple of the interloping lines $(i, j, k)$ where
none of $i$, $j$, or $k$ are equal (and in particular none of $z_i$, $z_j$, or
$z_k$ are equal). Now consider the case of removing emission from line $i$,
which is interloping at redshift $z_i$. One can cross-correlate line $i$ with
line $j$ if the frequency at which line $j$ is emitted at redshift $z_i$ is
within the observed frequency band. That is, we'd like to observe line $j$ at
a different frequency $\mynu{j}{o}'$ such that the emitted redshift $z_j'$ is
equal to $z_i$, so that we can cross-correlate the two. In other words, we
need $\mynu{j}{o}'$ to satisfy,
\beq\label{eq:nu_obs}
\begin{split}
\mynu{j}{o}' &= \mynu{i}{o} \frac{\mynu{j}{e}}{\mynu{i}{e}} \\
&= \mynu{i}{o} \frac{1+z_j}{1+z_i}\text{.}
\end{split}
\eeq
If the frequency $\mynu{j}{o}'$ is in the observing band of an instrument
(though in a different bin than $\mynu{i}{o}$), then a cross-corelation
between line $i$ and line $j$ can be made. This can be used to measure the
product $B_iB_j$. However, this does not allow us to remove either line $i$ or
line $j$ reliably, since the relevant quantity to subtract from
Equation~\ref{eq:observed_with_int} is $B_i^2 + B_j^2$, for which a direct
constraint cannot be made. Perhaps measurements of the cross-bispectrum may be
possible and allow us to measure $B_i$ and $B_j$ independently
\citep{2018ApJ...867...26B}, but measuring cross-bispectra is difficult in
practice and a reliable cross-spectrum measurement will probably be possible
with less survey area.

However, in the case that $\mynu{k}{o}'$ (defined similarly to
Equation~\ref{eq:nu_obs}) also lies in the observing band, then a three-field
approach can be employed to directly measure each of $B_i^2$, $B_j^2$, and
$B_k^2$.


\bibliography{references}

\end{document}
