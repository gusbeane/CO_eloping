\documentclass{aastex62}

\usepackage{graphicx}
\usepackage{gensymb}
\usepackage{bm}
\usepackage{amsmath}

\newcommand{\Gus}[1]{\textcolor{red}{#1}}

\newcommand{\Msun}{\ensuremath{\text{M}_\odot}}
\newcommand{\pc}{\ensuremath{\text{pc}}}
\newcommand{\kpc}{\ensuremath{\text{kpc}}}
\newcommand{\Mpc}{\ensuremath{\text{Mpc}}}
\newcommand{\Myr}{\ensuremath{\text{Myr}}}
\newcommand{\Gyr}{\ensuremath{\text{Gyr}}}
\newcommand{\SFR}{\ensuremath{\text{SFR}}}

\newcommand{\GHz}{\ensuremath{\text{GHz}}}

\newcommand{\CI}{\ensuremath{\text{C~\textsc{i}}}}
\newcommand{\CII}{\ensuremath{\text{C~\textsc{ii}}}}
\newcommand{\OIII}{\ensuremath{\text{O~\textsc{iii}}}}
\newcommand{\CO}{\ensuremath{\text{CO}}}

\newcommand{\abs}[1]{\left| #1 \right|}
\newcommand{\uth}{\textsuperscript{th}}
\newcommand{\n}{\text{n}}

\newcommand{\beq}{\begin{equation}}
\newcommand{\eeq}{\end{equation}}

\newcommand{\ps}[1]{\ensuremath{P_{#1,#1}}}
\newcommand{\xps}[2]{\ensuremath{P_{#1,#2}}}

\newcommand{\kmax}{\ensuremath{k_{\text{max}}}}

\newcommand{\mynu}[2]{\ensuremath{\nu_{#1,\text{#2}}}}
\newcommand{\denps}{\ensuremath{P_{\delta,\delta}}}
\newcommand{\pstot}[1]{\ensuremath{P_{#1,\text{tot}}}}
\newcommand{\Bsum}{\ensuremath{\tilde{B}}}

\newcommand{\E}[1]{\mathrm{E}[#1]}
\newcommand{\Var}[1]{\mathrm{Var}[#1]}
\newcommand{\Cov}[2]{\mathrm{Cov}[#1,#2]}
\newcommand{\avg}[1]{\ensuremath{\langle #1 \rangle}}
\newcommand{\SN}{\ensuremath{\text{S}/\text{N}}}

\newcommand{\penn}{Department of Physics \& Astronomy, University of
Pennsylvania, 209 South 33rd St., Philadelphia, PA 19104, USA}

\begin{document}

\title{Multi-line Intensity Mapping}

\email{abeane@sas.upenn.edu}

\author{Angus Beane}
\affil{\penn}

\author{Adam Lidz}
\affil{\penn}

\author{others}

\shorttitle{Multi-line Intensity Mapping}
\shortauthors{Beane et al.}

\begin{abstract}
Abstract.
\end{abstract}

\section{Introduction}\label{sec:intro}

\section{Three Field Approach}\label{sec:tf}
We begin by considering the three field approach. We first start with this
relatively simple case to gain some intuition for the \SN{} behavior when the
cross-spectra of multiple lines are combined.

We are concerned with the problem of estimating the power spectrum of each
line only from the three cross-spectra. To be more concrete, we model each
auto-spectrum and cross-spectrum as,
\beq\label{eq:auto_xps}
\begin{split}
\ps{i}(k,z) &= B_i(z)^2 \denps(k,z) \\
\xps{i}{j}(k,z) &= B_i(z) B_j(z) \denps(k,z)\text{,}
\end{split}
\eeq
where $B_i(z) \equiv b_i(z) \avg{I_i}(z)$. We will generally suppress the
redshift and $k$ arguments for brevity. We can model the auto-spectra and
cross-spectra in this way only on sufficiently large scales, i.e. only when
$k<\kmax$ for some value of $k_{\text{max}}$. The value of \kmax{} is set by
two conditions. First, we must have that linear biasing is a good description
for each field $i$. This way $B_i$ for each line will be independent of $k$.
Second, for each pair of lines $(i,j)$ we must have that $\abs{r_{i,j}(k)} =
1$ when $k<\kmax$.

Eventually it will be desirable to place constraints on $\denps(k)$. For now,
however, we simply wish to understand how well future experiments will be able
to determine each of $B_i$ as a function of $z$. We can proceed simply by
computing the relevant Fisher matrix:
\beq\label{eq:fisher_tf}
F_{i,j} = 
\int \frac{\text{d}k\,k^2}{2\pi^2} V_{\text{surv}} 
\sum_{\substack{l,l'\\l > l'}} \sum_{\substack{m,m'\\m > m'}}
\frac{\partial \hat{P}_{l,l'}(k)}{\partial B_i}
\frac{1}{\Cov{\hat{P}_{l,l'}(k)}{\hat{P}_{m,m'}(k)}} 
\frac{\partial \hat{P}_{m,m'}(k)}{\partial B_j}\text{,}
\eeq
where the sum of $l$, $l'$, $m$, and $m'$ is over each of the three lines.
Equation~\ref{eq:fisher_tf} can be readily modified to accomodate the case of
more than three lines (by extending the summations to include these lines) and
to accomodate the auto-spectra (by allowing $l=l'$ and $m=m'$ in the relevant
summations). We will consider these cases in Sections~\Gus{x}~and~\Gus{y},
respectively.

For the variance and covariance of the power spectra, it can be shown that,
\beq\label{eq:var_cov}
\begin{split}
\Var{\xps{i}{j}} &= \xps{i}{j}^2 + \pstot{i}\pstot{j} \\
\Cov{\xps{i}{j}}{\xps{i}{k}} &= \pstot{i}\xps{j}{k} +
\xps{i}{j}\xps{i}{k}\text{,}
\end{split}
\eeq
where $\pstot{i} \equiv \ps{i} + N_i$. Note that $N_i$ contains contributions
from instrument noise, residual foreground contamination, and
interloping/extraloping lines. While computing the Fisher matrix numerically
is trivial, we will first work out the noise-dominated case which has a
somewhat tractable analytic solution.

\subsection{White Noise-dominated Regime} \label{ssec:tf_noisedom}
In the white noise-dominated regime, we have that each $N_i \gg \ps{i}$ and
$N_i(k) = N_i$. The relevant variance and covariance formulae then reduce to,
\beq\label{eq:var_cov}
\begin{split}
\Var{\xps{i}{j}} &= N_iN_j \\
\Cov{\xps{i}{j}}{\xps{i}{k}} &= N_i\xps{j}{k}\text{.}
\end{split}
\eeq
Since the covariance formula is only first order in $N_i$, we will ignore
contributions from terms like it and only consider terms involving
$\Var{\xps{i}{j}}$. In such a case, the formula for the Fisher matrix
simplifies greatly and we can write it down as:
\beq \label{eq:fmat_tf_exp}
F = V_k^2
\begin{pmatrix}
\frac{B_2^2}{N_1N_2}+\frac{B_3^2}{N_1N_3} & \frac{B_1B_2}{N_1N_2} & \frac{B_1B_3}{N_1N_3} \\
\frac{B_2B_1}{N_2N_1} & \frac{B_3^2}{N_2N_3}+\frac{B_1^2}{N_2N_1} & \frac{B_2B_3}{N_2N_3} \\
\frac{B_3B_1}{N_3N_1} & \frac{B_3B_2}{N_3N_2} & \frac{B_1^2}{N_3N_1}+\frac{B_2^2}{N_3N_2}
\end{pmatrix}
\text{,}
\eeq
where,
\beq \label{eq:Vk}
V_k^2 \equiv V_{\text{surv}} \int \frac{\text{d}k\,k^2}{2\pi^2} \denps(k)^2\text{.}
\eeq

Inverting this matrix we find that the covariance matrix of the bias factors is,
\beq \label{eq:covmat_tf_wn}
C = F^{-1} = \frac{1}{4V_k^2}
\begin{pmatrix}
\frac{B_1^2N_2N_3 + B_2^2N_3N_1 + B_3^2N_1N_2}{B_2^2B_3^2} & \frac{-B_1^2N_2N_3 - B_2^2N_3N_1 + B_3^2N_1N_2}{B_1B_2B_3^2} & \frac{-B_1^2N_2N_3 + B_2^2N_3N_1 - B_3^2N_1N_2}{B_1B_2^2B_3} \\
\frac{-B_1^2N_2N_3 - B_2^2N_3N_1 + B_3^2N_1N_2}{B_1B_2B_3^2} & \frac{B_1^2N_2N_3 + B_2^2N_3N_1 + B_3^2N_1N_2}{B_3^2B_1^2} & \frac{B_1^2N_2N_3 - B_2^2N_3N_1 - B_3^2N_1N_2}{B_1^2B_2B_3} \\
\frac{B_1^2N_2N_3 - B_2^2N_3N_1 - B_3^2N_1N_2}{B_1^2B_2B_3} & \frac{B_1^2N_2N_3 - B_2^2N_3N_1 - B_3^2N_1N_2}{B_1^2B_2B_3} & \frac{B_1^2N_2N_3 + B_2^2N_3N_1 + B_3^2N_1N_2}{B_1^2B_2^2}
\end{pmatrix}
\text{.}
\eeq
Examining the diagonal components we have that:
\beq \label{eq:var_bias_tf_wn}
\begin{split}
C_{1,1} = \sigma_{B_1}^2 &= \frac{1}{4V_k^2} \left( B_1^2 \frac{N_2N_3}{B_2^2B_3^2} + N_1(\frac{N_2}{B_2^2} + \frac{N_3}{B_3^2}) \right) \\
\implies \frac{\sigma_{B_1}^2}{B_1^2} &= \frac{1}{4V_k^2} \left( \frac{N_1N_2}{B_1^2B_2^2} + \frac{N_2N_3}{B_2^2B_3^2} + \frac{N_3N_1}{B_3^2B_1^2} \right)\text{.}
\end{split}
\eeq
The symmetry of this result implies that,
\beq \label{eq:frac_error_same}
\frac{\sigma_{B_1}^2}{B_1^2} = \frac{\sigma_{B_2}^2}{B_2^2} = 
\frac{\sigma_{B_3}^2}{B_3^2}\text{.}
\eeq
Unfortunately the surprising result that the fractional error on each line is
the same only holds in the three field case.

Now we wish to consider how the \SN{} on each $B_i$ from the various
cross-spectra is related to the \SN{} in the auto-spectrum case. We first
write down the \SN{} from the auto-spectrum in the white-noise dominated
regime:
\beq \label{eq:autops_sn_wn}
\SN(B_1, \text{auto-spectrum}) = \frac{B_1^2}{N_1} V_k
\eeq

Now, let us compare the \SN{} of the two approaches in various cases. This
will be simpler to write if we introduce $\tilde{B}_i \equiv B_i^2/N_i$. Then,
\beq
\frac{\SN(B_1, \text{three-field})}{\SN(B_1, \text{auto-spectrum})} =
\frac{2}{\tilde{B}_1}
\sqrt{\frac{\tilde{B}_1\tilde{B}_2\tilde{B}_3}{\tilde{B}_1 + \tilde{B}_2 + \tilde{B}_3}}
\text{.}
\eeq
There are several interesting cases to consider with this formula. For
starters, let us assume that each field is measured to the same precision,
i.e. $\tilde{B}_1 = \tilde{B}_2 = \tilde{B}_3 = \tilde{B}$. In this case, we
have that
\beq
\frac{\SN(B_1, \text{three-field})}{\SN(B_1, \text{auto-spectrum})} =
\frac{2}{\sqrt{3}} \approx 1.15
\text{.}
\eeq
So, the three-field approach gives a marginally better \SN{} than the
auto-spectrum approach. This is encouraging, because it means that a tentative
auto-spectrum measurement could be checked against the value derived from the
three-field approach. Next, let us assume that the field $1$ has a much
stronger signal strength than $2$ or $3$ --- i.e. $\tilde{B}_1 \gg
\tilde{B}_2,\tilde{B}_3$. In this case we have that
\beq
\frac{\SN(B_1, \text{three-field})}{\SN(B_1, \text{auto-spectrum})} =
2\frac{\sqrt{\tilde{B}_2\tilde{B}_3}}{\tilde{B}_1}\text{.}
\eeq
In this case we see that it is unlikely for the $\text{S}/\text{N}$ on $B_1$
to be higher in the three-field case than in the auto-spectrum case. On the
other hand, let us consider the case where one of the other lines is measured
to much higher precision, e.g. $\tilde{B}_2 \gg \tilde{B}_1,\tilde{B}_3$. In
this case, we have that
\beq
\frac{\SN(B_1, \text{three-field})}{\SN(B_1, \text{auto-spectrum})} =
2\sqrt{\frac{\tilde{B}_3}{\tilde{B}_1}}\text{.}
\eeq
As long as field $1$ and $3$ are measured to comparable precision, a gain in
the \SN{} of $\sim2$ is possible with the three-field approach. Finally, let
us consider the most interesting case in which fields $2$ and $3$ are measured
to much higher precision than field $1$ --- i.e. $\tilde{B}_2, \tilde{B}_3 \gg
\tilde{B}_1$. For simplicity let us also assume that $\tilde{B}_2 \approx
\tilde{B}_3 = \tilde{B}$. In this case, we have that
\beq
\frac{\SN(B_1, \text{three-field})}{\SN(B_1, \text{auto-spectrum})} =
\sqrt{2}\sqrt{\frac{\tilde{B}}{\tilde{B}_1}}
\text{.}
\eeq
As a result, if two lines are measured with strong precision, then we can
strongly boost the \SN{} of a third, weakly measured line with the three-field
approach.

\subsection{\CO{} Forecasts for CCAT-p and CONCERTO} \label{ssec:forecasts_tf}
Armed with some intuition from the simple white noise-dominated case, we now
turn to forecasting constraints that can be placed on the CO bias factors. Our
modelling of the \CO{} intensity as a function of redshift follows
\citet{2016ApJ...825..143L} and is outlined in Appendix~\ref{app:co_int}. We
assume that $b_i=3$ for each \CO{} line.

\section{Thoughts}
Like doing a matched filter??? Trying to understand intuitively why this
works. If you have a super well measured field, then its like you're God who
knows that the density field is. Its almost like using that density field as a
matched filter?

\appendix
\section{CO Intensity Modelling} \label{app:co_int}
To model the statistical fluctuation and amplitude of the \CO{} intensity maps
we follow \citet{2016ApJ...825..143L}, and briefly summarize the procedure
here. We begin by assuming a Schechter form for the star formation rate
function,
\beq\label{eq:schechter}
\phi(\SFR)\,\text{d}\SFR = \phi_*
\left(\frac{\SFR}{\SFR_*}\right)^{\alpha}\exp{\left[-\frac{\SFR}{\SFR_*}\right]}
\frac{\text{d}\SFR}{\SFR_*}
\eeq
where $\alpha$ is the faint-end slope, $\SFR_*$ is the characteristic
star-formation rate, and $\phi_*$ is the characteristic number density. The
average specific intensity in a line $l$ can be related to the comoving
emissivity by \citep{2011ApJ...741...70L, 2013ApJ...768...15P},
\beq\label{eq:emiss_to_int}
\avg{I_l} = \frac{\epsilon_l}{4\pi \nu_{\text{rest},l}}\frac{c}{H(z)}\text{,}
\eeq
where $\nu_{\text{rest},l}$ is the rest-frame emission frequency and
$\epsilon_l$ is the comoving emissivity, each of line $l$. Assuming a linear
luminosity-SFR relation with a constant of proportionality of $L_0^l$, along
with the Schechter form for the SFR function (Equation~\ref{eq:schechter}), it
follows that \citep{2013ApJ...768...15P},
\beq\label{eq:com_emiss}
\epsilon_l = \phi_* L_0^{l}
\frac{\SFR_*}{1\,\Msun\text{yr}^{-1}}\Gamma{(2+\alpha)}\text{.}
\eeq
We adopt the value of $L_0^{\CO}$ for each line transition from
\citet{2010JCAP...11..016V}.


\bibliography{references}

\end{document}
