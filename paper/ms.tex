\documentclass{aastex62}

\usepackage{graphicx}
\usepackage{gensymb}
\usepackage{bm}
\usepackage{amsmath}

\newcommand{\Gus}[1]{\textcolor{red}{#1}}

\newcommand{\Msun}{\ensuremath{\text{M}_\odot}}
\newcommand{\pc}{\ensuremath{\text{pc}}}
\newcommand{\kpc}{\ensuremath{\text{kpc}}}
\newcommand{\Mpc}{\ensuremath{\text{Mpc}}}
\newcommand{\Myr}{\ensuremath{\text{Myr}}}
\newcommand{\Gyr}{\ensuremath{\text{Gyr}}}

\newcommand{\CII}{C~\textsc{ii}}

\newcommand{\abs}[1]{\left| #1 \right|}
\newcommand{\uth}{\textsuperscript{th}}
\newcommand{\n}{\text{n}}

\newcommand{\beq}{\begin{equation}}
\newcommand{\eeq}{\end{equation}}

\newcommand{\mynu}[2]{\ensuremath{\nu_{#1,\text{#2}}}}
\newcommand{\denps}{\ensuremath{P_{\delta,\delta}}}
\newcommand{\xps}[2]{\ensuremath{P_{#1,#2}}}
\newcommand{\pstot}[1]{\ensuremath{P_{#1,\text{tot}}}}

\newcommand{\E}[1]{\mathrm{E}[#1]}
\newcommand{\Var}[1]{\mathrm{Var}[#1]}
\newcommand{\Cov}[2]{\mathrm{Cov}[#1,#2]}
\newcommand{\avg}[1]{\ensuremath{\langle #1 \rangle}}

\begin{document}

The primary data is the observed intensity in a collection of voxels. In
practice, one will make measurements in sky coordinates and frequency space,
but to simplify the discussion we consider the observation of a finite
comoving box with volume $V$. The box is small enough that if one length is
assumed to lie along frequency space that the corresponding redshift extent of
that axis is small enough that the observed voxels can be considered
co-evolving. We are then interested in measuring $I_{l}(\bm{x})$, or the
collective emission from line $l$ from each voxel. In order to understand the
statistical properties of $I(\bm{x})$, we can compute the power spectrum which
is related to the Fourier transform of $I$,
\beq
P_{l}(k) = \avg{I(\bm{k}) I^{*}(\bm{k})}\text{.}
\eeq
Since the line intensity is a biased tracer of the underlying density field,
it is useful to parametrize the power spectrum as,
\beq
P_l(k) = b_l^2 \avg{I_l}^2 \denps{P_{\delta}}(k)\text{,}
\eeq
where $\denps{P_{\delta}}$ is the matter density power spectrum. Measuring
$b_l$ and $\avg{I_l}$ independently is a useful endeavour, but not the focus
of our study. Therefore, we will simplify and define $B_l \equiv b_l
\avg{I_l}$, such that,
\beq
P_{l}(k) = B_l^2 \denps(k)
\eeq

Suppose we have a line $l$ which we are attempting to measure emission from at
some target redshift $z_l$ and some observed frequency $\mynu{l}{o} =
\mynu{l}{e}/(1+z_l)$. Then, the interloper problem is that at various other
redshifts $z_i$ for $i=1,2,3,...$ there will be emission from lines with
emitted frequencies at redshift $z_i$ that satisfy the constraint $\mynu{l}{o}
= \mynu{i}{o}$, or equivalently,
\beq
\mynu{i}{e} = \mynu{l}{e}\frac{1+z_i}{1+z_l}\text{.}
\eeq


\bibliography{references}

\end{document}
